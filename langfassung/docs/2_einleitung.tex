Nach meinem letzten Jugend-Forscht Projekt ergab sich mir die Möglichkeit ein
Praktikum im Zentrum für Astronomie in Heidelberg zu absolvieren. Über die
Social-Media Plattform Reddit stellte ich den Kontakt mit Tim Tugendhat her
der zurzeit seinen PhD. in Physik an der Universität in Heidelberg macht.
Dieser ermöglichte es mir, die physikalische Fakultät an einer Uni mal genauer
zu sehen und das täglich leben eines Physikers mitzuerleben.
\par
Während des Praktikums stellte ich fest das ich die im letzten Jahr erlerne Fähigkeit mit
Python\footnote{Programmiersprache} zu Programmieren und mit
Blender\footnote{3D Software Suite} umzugehen nutzen konnte um Galaxien
darzustellen.
Dies war insgesamt unglaublich interessant und zeigte mir zum wiederholten mal:
Projekte sind sehr dazu geeignet um sich in neues einzuarbeiten oder neues
zu lernen und bieten einem ein Ziel welches man erreichen möchte was einem
immer genügend Motivation bietet weiterzumachen.
\par
Eine frage die ich mir öfters gestellt habe war, warum man eigentlich Galaxien
simuliert? Wäre es nicht einfacher einfach in den Himmel zu gucken und
die bereits bestehenden Galaxien zu beobachten?
Nach kurzer Recherche lag die Antwort auf der Hand: Galaxien brauchen mehrere
Millionen Jahre um sich zu entwickeln, also kann man ihre Entwicklung als
normaler Mensch nicht in dem Umfang beobachten, um dann daraus Schlüsse zu
ziehen. Daher simuliert man die Galaxien und kann dann somit vorhersagen oder
herausfinden wie die Galaxien entstanden sind bzw. was mit ihnen passieren
wird.

\subsection{Themen}

\begin{itemize}
  \item Generierung von elliptischen Galaxien
  \item Generierung von einem Dark-Matter Halo um die elliptische Galaxie
  \item Stauchung und Streckung des Dark-Matter mit Beeinflussung der eigentlichen Galaxie
  \item Beschleunigung des Generierungsprozesses mithilfe einer sogenannten ''lookup-table''
  \item Aufbau eines neuronalen Netzes für die unbeaufsichtigte Generation von Galaxien
  \item Generation von Spiralgalaxien
\end{itemize}

\subsection{Motivation}

Die Motivation für das Projekt kam praktisch direkt nach meinem letzten Jugend
Forscht Projekt bei dem ich mich mit der Vorhersage von Satellitenkollisionen
beschäftigt habe. Durch mein Praktikum im Zentrum für Astronomie der Uni Heidelberg
kam ich auf die Idee, ich könnte mein Wissen im Bezug auf die Programiersprache
Python und der 3D-Suite Blender mithilfe eines Projektes erweitern.
Ein Projekt zu haben um sich an etwas Neues heranzuwagen ist sehr empfehlenswert
wie ich schon in letzten Jahr gemerkt habe, ich hatte also wieder ein Projekt
welches mich Tag für Tag motiviert hat etwas zu erreichen.
