Nach meinem letzten Jugend-Forscht Projekt ergab sich die Möglichkeit ein
Praktikum im Zentrum für Astronomie in Heidelberg zu absolvieren. Über die
social-media Platform Reddit stellte ich den kontakt mit Tim Tugendkat her
der zurzeit seinen PhD. in Physik an der Universität in Heidelberg macht.
Dieser ermöglichte es mir, die Physikalische Fakultät an einer Uni mal genauer
zu sehen und das täglich leben eines Physikers mitzuerleben.
\par
Während des Praktikums stellte ich fest das ich die im letzten Jahr erlerne Fähigkeit mit
Python\footnote{Programmiersprache} zu Programmieren und mit
Blender\footnote{3D Software Suite} umzugehen nutzen konnte um Galaxien
darzustellen.
Dies war insgesamt unglaublich Interessant und zeigte mir zum wiederholten mal:
Projekte sind sehr dazu geeignet um sich in neues einzuarbeiten oder neues
zu lernen und bieten einem ein Ziel welches man erreichen möchte was einem
immer genügend motivation bietet weiterzumachen.
\par
Eine frage die ich mir öfters gestellt habe war warum man eigentlich Galaxien
simuliert? Wäre es nicht einfacher einfach in den Himmel zu gucken und
die bereits bestehenden Galaxien zu beobachten?
Nach kurzer recherche lag die Antwort auf der Hand: Galaxien brauchen mehrere
Millionen Jahre um sich zu entwickeln, also kann man ihre Entwicklung als
normaler Mensch nicht in dem Umfang beobachten, um dann daraus schlüsse zu
ziehen. Daher simuliert man die Galaxien und kann dann somit vorhersagen oder
herrausfinden wie die Galaxien entstanden sind bzw. was mit ihnen passieren
wird.

\subsection{Themen}

\begin{itemize}
  \item Generierung von Elliptischen Galaxien
  \item Generierung von einem Dark-Matter Halo um die Elliptische Galaxie
  \item Stauchung und Streckung des Dark-Matter mit beinflussung der eigentlichen Galaxie
  \item Beschleunigung des generierungsprozesses mithilfe einer sogennanten ''lookup-table``
  \item Aufbau eines neuronalen Netzes für die unbeaufsichtigte Generation von Galaxien
  \item Generation von Spiralgalaxien
\end{itemize}

\subsection{Motivation}

Ich habs einfach mal getan...
