\begin{abstract}
% \large
Das Ziel meines Projektes ist es, Realitätsgetreue Galaxien und Dunkle Materie
Halos zu generieren.
Hierzu verwende ich das sogenannte ''Navarro-Frenk-White'' Profil welches in
Kombination mit der ''Random Sampling'' Methode die Dichteverteilung
der Sternenpositionen in Koordinaten für einzelne Sterne umgewandelt.
\par
Vergleicht man die generierten Galaxien mit echten Galaxien fällt auf das
die Sterne sich anders verhalten. Dies lässt sich durch Dunkle Materie erklären,
welche man jedoch nicht direkt beobachten kann. Es kann also
nur aufgrund ihrer Auswirkungen auf andere Objekte auf sie geschlossen werden,
weshalb es nicht ganz Trivial ist sie sichtbar darzustellen.
\par
Im Verlauf des Projektes haben sich mir jedoch auch andere Teilbereiche
eröffnet wie z. B. die Generation von Spiralgalaxien, die Optimierung von
Rechenprozessen und die Nutzung von einem neuronalen Netz zur Anpassung der
generierten Galaxie an eine reale Galaxie.
\end{abstract}
