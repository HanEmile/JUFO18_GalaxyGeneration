Quellen

\begin{center}
 \textbf{
  Das Python-Programm sowie die Blender Darstellungen wurden vollständig ohne fremde Hilfe selber erstellt.
 }
\end{center}
\par Einen Großteil der Formeln fand ich durch eine Wikipedia Recherche, jedoch wurden auch Informationen aus dem 'SPACETRACK REPORT' von Felix R. Hoots und Ronald L. Roerich,
No. 3 (Dec. 1980) entnommen.
\par Das Programieren in der Programiersprache Python habe ich wärend des Projektes mithilfe der Python-Documentation gelernt. Mit dem Umgang des 3D-Programms Blender
bin ich schon vertraut gewesen. Die Grundlagen für \LaTeX, in dem diese Langfassung geschrieben wurde, erlernte ich durch das Studieren diverser Beiträge in Foren und der Einsicht
in das Jugend Forscht Projekt von Konstantin Bosbach, Tilman Hoffbauer und Steffen Ritsche aus dem vorherigem Jahr (2016, Underwater Accoustic Communication).
\par Da es das erste Mal war, dass ich Python mit Blender kombiniert habe, musste ich mich in diesen Bereich ebenfalls einarbeiten.
Dabei war die Blender API Documentation (www.blender.org/api) von großer Hilfe.


\raggedleft
\section*{Dank gilt...}
\paragraph{Herrn Jörg Thar} meinem Betreuer
\paragraph{Konstantin Bosbach} welcher mir eine Möglichkeit gab für 2 Wochen in Heidelberg zu wohnen.
\paragraph{Tilman Hoffbauer}

\centering
\vspace{0.5cm} \textbf{Außerdem gilt mein Dank allen, die mich auf jede nur erdenkliche Weise unterstützt haben.}
