\begin{center}
 \textbf{
  Das Python-Programm sowie die Blender Darstellungen wurden vollständig ohne fremde Hilfe selber erstellt.
 }
\end{center}
\par Einen Großteil der Formeln fand ich durch eine Wikipedia Recherche.
\par Das Programieren in der Programiersprache Python habe ich während meines Jugen-Forscht Projektes im letztem Jahr (2017) gelernt. Mit dem Umgang des 3D-Programms Blender
bin ich schon vertraut gewesen. Die Grundlagen für \LaTeX, in dem diese Langfassung geschrieben wurde, erlernte ich durch das Studieren diverser Beiträge in Foren und der Einsicht
in das Jugend Forscht Projekt von Konstantin Bosbach, Tilman Hoffbauer und Steffen Ritsche (2016, Underwater Accoustic Communication).
Die Einführung in die Mathematik bekam ich während meines Praktikums im Zentrum Für Astronomie in Heidelberg durch Tim Tugendhat.
\raggedleft
\section*{Dank gilt...}
\paragraph{Herrn Jörg Thar} meinem Betreuer
\paragraph{Tim Tugendhat} der mir es ermöglichte ein Praktikum im Astronomischen Recheninstitut zu machen.
\paragraph{Konstantin Bosbach} welcher mir eine Möglichkeit gab für 2 Wochen in Heidelberg zu wohnen.
\paragraph{Tilman Hoffbauer} der bei Problemen bereit war Licht ins Dunkle zu bringen.


\centering
\vspace{0.5cm} \textbf{Außerdem gilt mein Dank allen, die mich auf jede nur erdenkliche Weise unterstützt haben.}
