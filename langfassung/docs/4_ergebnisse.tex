\subsection{Simulations Geschwindigkeit}

Insgesamt bin ich zu folgenden Ergebnissen zusammengekommen:

\begin{itemize}
  \item
  \begin{tabular}{l | l | l}

    Sterne  & Zeit (30.10.2017)  & Zeit (\today) \\ \hline\hline
    45000     & ca. 9h            & ca. 4h \\
  \end{tabular}

  \item Pro MegaByte können die Koordinaten von 10000 Sternen gespeichert werden.

  \item Die Nutzung von Lookuptabellen ist unglaublich sinnvoll
\end{itemize}

\subsection{Lookuptabellen Geschwindigkeit}

\begin{tabular}{l | l | l}
Rho-werte  & Schrittweite & Zeit zum einlesen (in sekunden) \\ \hline\hline
1500000     & 1     & 8.07  \\
750000      & 2     & 4.4   \\
375000      & 4     & 2.26  \\
187500      & 8     & 1.35  \\
93750       & 16    & 0.76  \\
\end{tabular}

Hier ist klar zu sehen, dass es eine lineare korrelation zwischen der Anzahl
an generierten Werten und der Zeit gibt.

\subsection{Fazit}

Insgesamt betrachtet kann ich behaupten das das Projekt ein voller Erfolg war:
Ich habe unglaublich viele neue sachen gelernt und dabei ein Funktionierendes
Program zur visualisierung von Galaxien und DUnkler Materie Gebaut. Dabei bekam
ich einblicke in die verschiedensten Teilgebiete der Physik, AstroPhysik, Matematik
und Informatik. Viele dieser Themen konnte ich jedoch aufgrund ihrer Komplexität
nur in geringen maßen nutzen, weshalb ich mich in der Zukunft gerne damit
auseinander setzen würde. Ein Beispiel hierfür sind die Neuronalen Netze welche
ein enormes Potential haben welches ich unbedingt nutzen möchte.
