Ergebnisse

\subsection{Simulation Speed}

Nach mergen des speed-branches sind folgende Ergebnisse zusammengekommen:

\begin{tabular}{l | l | l}

Sterne  & Zeit Vorher   & Zeit Nachher \\ \hline\hline
1e5     & 2.93 sek.     & k.A. \\
1e6     & 29.38 sek.    & k.A. \\
1e7     & 315.67 sek.   & k.A. \\
1e9     & 9h            & k.A. \\

\end{tabular}

Aus 1e9 Sternen werden vorher letztendlich 45000 Sterne generiert.

Pro MegaByte können die Koordinaten von 10000 Sternen gespeichert werden.

\subsection{Spiral Galaxies}

Die generierung von Spiralgalaxien gestaltet sich schweiriger als erwartet.

\subsection{Lookup-Table Speed}

\begin{tabular}{l | l | l}
rho-values  & step  & time (in seconds) \\ \hline\hline
1500000     & 1     & 8.07  \\
750000      & 2     & 4.4   \\
375000      & 4     & 2.26  \\
187500      & 8     & 1.35  \\
93750       & 16    & 0.76  \\
\end{tabular}

The correlation between the number of stars generated and the time needed ist
clearly linear.

\paragraph{Python script} ~\\
\lstset{language=Python}
\begin{lstlisting}[frame = single]
import matplotlib.pyplot as plt

list_time = [8.07, 4.4, 2.26, 1.35, 0.76]
list_rho_values = [1500000, 750000, 375000, 187500, 93750]

plt.plot(list_time, list_rho_values, '-ro')
plt.show()
\end{lstlisting}

\subsection{Distortion of Galaxies}

Galaxien verformen dinge
