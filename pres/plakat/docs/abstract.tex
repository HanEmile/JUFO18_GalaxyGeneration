\begin{abstract}

  Das Ziel meines Projektes war es, Galaxien drei-dimensional zu
  visualisieren.
  Dazu verwendete ich das sogenannte ''Navarro-Frenk-White''' Profil als
  Dichtefunktion in Kombination mit der Random-Sampling Methode zum Generieren
  von Koordinaten.
  \par
  Die Generierung der Koordinaten hat jedoch ein Problem: Es ist extrem
  rechenaufwendig. Um dieses Problem zu lösen, habe ich mehrere Methoden zur
  Optimierung angewendet, darunter die Nutzung von Lookup-Tabellen und die
  Verwendung von mehreren Computer-Rechenkernen.
  \par
  Im Verlauf des Projektes habe ich viel Neues dazugelernt, darunter die
  Handhabung mit großen Datenmengen und die Organisation eines größeren Projektes,
  jedoch auch mit vielen tiefer gehenden Funktionen und Bibliotheken
  in der Programmiersprache Python umzugehen.

\end{abstract}


\vspace{0.5cm}
\setlength{\fboxsep}{10pt}
\fbox{
  \parbox{0.95\linewidth}{
  An das Praktikum im Astronomischen Recheninstitut in Heidelberg gelangte
  ich, indem ich mein Jugend Forscht Projekt aus dem letztem Jahr (Satellite
  Computation), auf der Social-Media Plattform Reddit hochgeladen habe, wonach
  mich der Doktorand Tim Tugendhat anschrieb, ob ich Interesse an einem
  Praktikum hätte, das ich im Sommer 2017 absolviert habe.
  }
}
