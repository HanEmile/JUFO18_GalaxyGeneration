\documentclass[aspectratio=169]{beamer}
\usepackage[utf8]{inputenc}
\usepackage[T1]{fontenc}

\usetheme{Singapore}

\title{Galaxy Generation}
\subtitle{Jugend Forscht 2018}
\author{Emile Hansmaennel}
\institute{Theodor Fliedner Gymnasium}
\date{\today}


\begin{document}
  \begin{frame}
    \titlepage
  \end{frame}

  \begin{frame}
    \tableofcontents
  \end{frame}

  \section{Point Cloud}
  \subsection{Random generation of points}

  \begin{frame}
    \frametitle{Generating an evenly distributed point-cloud}

    \begin{itemize}
      \item numpy
    \end{itemize}

  \end{frame}

  \subsection{Random-Sampling with the NFW-profile}

  \begin{frame}
    \frametitle{Random Sampling}

    \begin{itemize}
      \item Generate a random value in a range \( [r_{min};r_{max}] \)
      \item Find out if the value is bigger or smaller than the NFW value
    \end{itemize}

  \end{frame}

  \begin{frame}
    \frametitle{The NFW-Profile}

    \begin{itemize}
      \item Returns a probability for a star to get generated
    \end{itemize}

    [insert nfw-distance function image]
  \end{frame}

  \subsection{Saving the stars}

  \begin{frame}
    \frametitle{Using the .csv file format}

    \begin{itemize}
      \item How do I use the csv format?
    \end{itemize}
  \end{frame}

  \section{Forces}
  \subsection{Generating a grid for subdividing the galaxy into cells}

  \begin{frame}
    \frametitle{Generating a grid for subdividing the galaxy}

    \begin{itemize}
      \item insert grid image without the spheres
      \item problem (threshold)
    \end{itemize}
  \end{frame}

  \subsubsection{Generating the spheres}

  \begin{frame}
    \frametitle{Generating the spheres on the vertices of the grid}

    \begin{equation}
      r = \sqrt{1^2 + 1^2 + 1^2}
    \end{equation}

    \begin{itemize}
      \item insert grid image with the spheres
      \item problem (threshold)
    \end{itemize}

  \end{frame}

  \subsubsection{Finding out which star is in which spheres}

  \begin{frame}
    \frametitle{Which Star is in which sphere?}

    \begin{itemize}
      \item Method for finding this out
    \end{itemize}

  \end{frame}

  \subsection{Calculate the forces acting inside of each sphere}

  \begin{frame}
    \frametitle{Forces acting in each sphere}

    \begin{itemize}
      \item cycle through all the stars in each sphere and calculate the forces
    \end{itemize}
  \end{frame}

  \subsubsection{Calculation the Forces acting between one star and the rest}

  \begin{frame}
    \frametitle{Calculate the forces acting inbetween the individual stars
    and the other stars}

    \begin{itemize}
      \item DO EPIC THINGS!
    \end{itemize}
  \end{frame}

  \section{Extrapolation}
  \subsection{Calculate where the stars are next}

  \begin{frame}
    \frametitle{Where are the stars after n seconds?}

    \begin{itemize}
      \item Force acting on star for n seconds equals a displacement of m
    \end{itemize}
  \end{frame}

\end{document}
